\chapter{Introduction}
It seems pretty common that when we think about software development, the thoughts usually wander towards different kinds of agile principles, object-oriented programming, functional programming, design patterns and so on. Whatever the case may be, the role of usability seems to lie somewhere higher on the hierarchy of the creation process. That is to say, usability is not necessarily involved in the process from the ground up, and this can have a detrimental effect on the performance of the software. No matter how powerful the underlying solutions might be, it is still usually people using it to the best of their ability, trying to take advantage of the underlying implementation. It is important to present information in a usable way, so it can be utilized in the best possible way. Also, people have become more usability conscious, partly due to the rise of smartphones and other touch devices \cite{Chen2015}. Usability is important also for a type of software called video games, and the type of usability evaluation methods that could or should be used in game development is a debated subject: different evaluation methods exist, but which are best for game design---also given there are so many different types of games and genres---is not always clear \cite{Bernhaupt2007}. Even though games are mostly seen as entertainment, bad games are hardly entertaining. The potential frustration of the player can be countered with properly guiding the player into the game, and making sure they learn the necessary properties and mechanics of the game to acquire sufficient mastery of the system in order to experience it in a meaningful way. This seems important also because it has been found that many players give up on a game during the very first hours \cite{Bauckhage2012}. One hypothesis then is that in order to make the player stick with the game we have to make sure the usability factor is not ignored during these first hours. The understanding of both gameplay and narrative are important factors in player retention \cite{Cheung2014}. A common theme for many games is that they introduce a tutorial at the beginning of the game, which aims---or at least should aim---to teach the basic mechanics and necessary interface elements, anything that is fundamental to the basic gameplay. 

The aim of this thesis is to provide a heuristic framework for performing an expert review on video game tutorial usability, based on selected video game tutorials and existing research on game usability heuristics, as opposed to a usability evaluation with a group of test users. As mentioned before, the first few hours anyone spends with a game can be critical for player retention, so the time bracket for the potential of tutorials as something that will encourage the player to keep going seems significant, as the tutorial is usually the first thing the player encounters in a game. Also, games are a recreational activity which makes them easier to abandon. If a game does not seem interesting enough at first, it will not matter how much fun it is later on, if the player quits early. \cite{Cheung2014}. A tutorial, then, should be a representative of the game.

This thesis has the following structure: In Chapter 2, we go over the concept of usability from a general level to usability in software and video games. We then talk about what are heuristics, and why there are tutorials. Chapter 3 dives into heuristics and tutorials in more detail, and we define a set of relevant existing heuristics to start our analysis with. In Chapter 4, we then go over the tutorials of our chosen games, and attempt to expand on the heuristics we have listed so far to find improvements and additional heuristics to help with tutorial usability. The findings are presented in Chapter 5 and the topics further discussed in Chapter 6.