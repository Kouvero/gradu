\chapter{Introduction}
It seems pretty common that when we think about software development, the thoughts usually wander towards different kinds of agile principles, object-oriented programming, functional programming, design patterns and so on. Whatever the case may be, the role of usability seems to lie somewhere higher on the hierarchy of the creation process. That is to say, usability is not necessarily involved in the process from the ground up, and this can have a detrimental effect on the performance of the software, no matter how powerful the underlying solutions might be; it is still usually people using it to the best of their ability, trying to take advantage of the underlying implementation. Because information is communicated to the user, the way things are presented matters. This is important also for a type of software called video games. Even though games are mostly seen as entertainment, bad games are hardly entertaining. The potential frustration of the player can be countered with properly guiding the player into the game, and making sure they learn the necessary properties and mechanics of the game to acquire sufficient mastery of the system in order to experience it in a meaningful way. This seems important also because it has been found that many players give up on a game during the very first hours \cite{Bauckhage2012}. One hypothesis then is that in order to make the player stick with the game we have to make sure the usability factor is not ignored during these first hours. Both the understanding of gameplay and narrative are important factors in player retention \cite{Cheung2014}. A common theme for many games is that they introduce a tutorial at the beginning of the game which aims --- or at least should aim --- to teach the basic mechanics and necessary interface elements, anything that is fundamental to the basic gameplay. 

The aim of this thesis is to provide an expert review of selected video game tutorials, as opposed to a usability evaluation with a group of test users. A hypothesis for whether video game tutorials are usable does not feel intuitive as such, i.e. it could go either way, and of course depends on the game in question (some tutorials probably are usable and some are not). Based on some research, the first few hours anyone spends with a game can be critical for player retention, so the time bracket for the potential of tutorials as something that will encourage the player to keep going seems significant: "The first time a player sits down with a game is critical for their engagement. Games are a voluntary activity and easy to abandon. If the game cannot hold player attention, it will not matter how much fun the game is later on if the player quits early." \cite{Cheung2014}.