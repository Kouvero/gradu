\chapter{Analyzing expert review results}

How to combine our experience with the games in the previous chapter with the heuristics we looked at earlier? Since there are games from various genres it is justified to have a somewhat abstract take on how they are or should be presented. It also depends a lot on the context, since principles used in a good fighting game tutorial do not necessary apply to a game of another genre that well. This means that when thinking about how to build a good tutorial, you only need some tools from the toolbox, but a versatile toolbox is still a good thing to have. If there is no hammer, it does not mean that nails are not a good thing. This is also partly the case with \textit{Minecraft: Pocket Edition}. If there was a tutorial, the game might turn out to be easy and fun to play, but the lack of one demonstrates that good and usable things are not necessarily intuitive at first \cite{Raskin1994}. This is the reason it is important to teach the player, no matter how simple the controls would be. One related example is text and programming editors such as Vim. They are not intuitive to use and require a long process to learn and modify to your own purposes, but eventually large boosts in productivity may be experienced \cite{Robbins2008}. However, games are in most cases a voluntary pursuit for the player, not a tool, and are works of imagination and art to be enjoyed mostly as entertainment in various genres, so making a one size fits for all collection of heuristics is rather impossible. Similarly, one could also say that the game should not be a chore to learn, no matter how fun it might become after hours of learning. It is important to have tools to analyze tutorials and games, but games that in itself feel like a chore to the player rarely succeed. In other words, productivity software shoud aim to make things easy and efficient, whereas games should be fun and challenging enough \cite{Pagulayan2003}. 

One central thing emerges from the concept of a tutorial: it does not generally matter if you have a separate tutorial, or a sort of in-game help and instruction that guides the player in to the game. This can even happen for a very long period of time, as a game gradually introduces new concepts as the characters develop and, for example, gain new abilities for hours into the game, up until the very last levels even. It is not always just "learn everything and start playing". The game design choice of having mechanics come up later in the game means that it is not necessarily a good idea to teach all those mechanics beforehand in a tutorial at the very beginning of the game. Having small \textit{separate} tutorials along the game could be distracting in the long run, so it is better to embed them within the actual game as new properties are unlocked. The tutorial, then, becomes a little ambiguous as a concept. A game might instruct the player really well but not have a separate tutorial to do that, or the game might have a separate tutorial and also instruct the player really well in that as well. It is safe to say that when we are talking about a tutorial, we can talk about the way the game attempts to transfer the required knowledge to the player, as we defined tutorials in an earlier chapter. A common nominator is that the tutorial should begin with things of lower complexity and then move on to higher level concepts. This is also called \textit{priority learning} \cite{Bycer2016}. With these results in mind, in the following section we will go through the game usability heuristics found in the literature and discussed before, and attempt to enhance our view on how video game tutorials can be analyzed regarding their usability based on the games we discussed.

\section{Tutorials and existing heuristics}
\paragraph{1. Easy to learn, hard to master.} How can we say that a tutorial should be easy to learn, but hard to master? In the case of games like \textit{Dark Souls 2}, the presence of a tutorial might not be completely evident, but it is still not a very difficult tutorial. If anything, if there is a tutorial, the whole point of it is to transfer information. This means that if there is one, it should do its job well, or not exist at all, like with \textit{This War of Mine}. The lack of a tutorial in \textit{Minecraft: Pocket Edition} is different, because there is nothing on the screen that tells us that there are possible actions, unlike in \textit{This War of Mine}. Using this heuristic for a tutorial, we can just say that the game---or rather the tutorial---should be \textit{easy to learn}.

\paragraph{2. The goals are clear. } This heuristic applies to tutorials as well. They should be clear about what they aim to teach and what the player's next action should be if applicable.

\paragraph{3. The skills needed to attain goals are taught early enough to play or use later, or right before the new skill is needed.} When the game design is such that all of the possible actions are not immediately accessible, like in \textit{Ori and the Blind Forest}, it is a good idea give that information to the player only right before the new mechanic becomes available to the player. 

\paragraph{4. The first ten minutes of play and player actions are painfully obvious and should result in immediate and positive feedback for all types of players. } This seems like something that is present in all of the games except \textit{Minecraft: Pocket Edition}, which shows that this lack of information can have a detrimental effect on learning the game even with a seemingly simple game.

\paragraph{5. Player does not need to read the manual or documentation to play.} What is documentation? If there is a tutorial that has text in it, even if presented in small parts, does it not essentially mean that the player is reading the documentation? Accessing a separate document or file, however, is not necessarily desirable. Also, there are complicated genres like flight simulators where a huge documentation is a necessity. Having all that information presented in-game in a fluid manner would still be a good thing, so this heuristic is a good goal in that sense, but not always practical.

\paragraph{6. Player does not need to access the tutorial in order to play.} The ability to skip a tutorial is an important one, just design it so that someone playing the game for the first time does not do accidentally.

\paragraph{7. Status score indicators are seamless, obvious, available and do not interfere with game play.}
This becomes meaningful in, for example, sequenced tutorials where the next action is not available until the current one has been completed, so it is important to communicate that necessity properly.

\paragraph{8. The game provides feedback and reacts in a consistent, immediate, challenging and exciting way to the player's actions.} Like in the previous point, making clear that the player just did the correct required action is important. The \textit{Guilty Gear Xrd Revelator} tutorial is a great example of this.

\paragraph{9. Provide appropriate audio/visual/visceral feedback (music, sound effects, controller vibration.} This, again, adds to the previous point in question. It should just be considered that there are multiple ways to communicate a success simultaneously, that is, not just in text but also in color, sound and, for example, controller vibration.

\paragraph{10. The player is given controls that are basic enough to be learned quickly, yet expendable for advanced options for advanced players.} This is very true with a game like \textit{Guilty Gear Xrd Revelator}. The fact that even the tutorial in the game goes in that depth to demonstrate advanced combinations extended from the basic controls makes this an important heuristic to have.

\paragraph{11. Player interruption is supported, so that players can easily turn the game on and off and be able to save the game in different states.} \textit{Insurgency} has a moderately long tutorial section, and it had a tendency to crash a number times while playing it. This meant that the tutorial had to be started again from the beginning which was frustrating. Adding a save game option to be used from a separate tutorial context is thus a good idea.

\paragraph{12. Upon turning on the game, the player has enough information to begin play.} This is self-evident in a way. If there is a separate tutorial, it is a good idea to have it accessible from the same menu as the main game, so that the option is clear.

\paragraph{13. Players should be given context-sensitive help while playing so that they are not stuck and need to rely on a manual for help.} 
In a tutorial it should always be clear what the next expected player action is.

\paragraph{14. All levels of players are able to play and get involved quickly and easily with tutorials and/or progressive or adjustable difficulty levels. }
Here, this essentially means that the tutorial should be easily accessible if there is a separate one, or that the in-game tutorial should be good enough according to the other heuristics. In general, it should not take too long for player to be able to actually do something in the game, which is mostly likely some part of the tutorial.
 
\paragraph{15. Get the player involved quickly and easily.}
This is practically same as the previous one. With the core idea being the tutorial, it does not need to be separately mentioned in the later heuristics. Also, as the heuristics regard tutorials, it can be assumed that the tutorial is already in progress, so how quickly the player gets to the tutorial is not in essence the problem. It does become a problem when there is a lack of instruction or a tutorial altogether, like in \textit{Minecraft: Pocket Edition}. But since we are evaluating the usability of a tutorial, we can assume that there already is a tutorial we are evaluating.

\paragraph{16. The game should give hints,  but not too many.} For progress in the actual game this is likely a good idea, but a tutorial should not give hints. It should be as clear as possible, not "maybe you need to press A to jump, maybe B, who knows...". This might still be enjoyable if humor was the intent, and rules are meant to be broken anyway.

\paragraph{17. Allow users to skip non-playable and frequently repeated content.} This is very important for a tutorial too, just so that it is not done accidentally.

\paragraph{18. Provide instructions, training and help.} These have all been mentioned earlier, and are in a way the essence of tutorials. Training, like it is in \textit{New Star Soccer} or \textit{The Banner Saga}, is a great way to enhance the gaming experience in addition to other basic information that would be considered a tutorial. To be able to practise on your own, not in a guided setting and not having to worry about the outcome too much, can be beneficial.

\section{Heuristics for tutorials}
Now that we have distilled our experience with the games in question through the heuristics used with general game design and usability found from the literature, it is time to think about heuristics for video game tutorials specifically. One difficult thing with this is the wording of heuristics. We even found articles listing ways to make a bad tutorial (many of which are surprisingly close opposites of our results) \cite{Adams2011}. As discussed earlier, not all heuristics need apply in order for the tutorial to be efficient. It also depends a lot on the genre of the game for example. There is a tendency with heuristics to have a sort of a "you should do this" type of tone, whereas in practise, they are only things to be considered that might fit the game's needs or not. We need to think about how to present the information so that everything does not come across as a necessity, but rather a point to consider. Not all games necessarily need a training mode for example, so it can be problematic to say "have a training mode". Using the takeaways from the experiences with the games earlier and the general heuristics discussed in the context of tutorials, we can combine them to create a list of heuristics that is more applicable to the usability of video game tutorials specifically. We were able to identify and create a total of 27 heuristics---also available in appendix A---which are as follows:

\paragraph{1. The tutorial should reflect the intended gameplay experience.} 
This can be a conflicting heuristic, since with a game like \textit{Dark Souls 2} the question arises "should we also make the tutorial unforgiving or difficult, similar to the core gameplay experience". In the case of \textit{Dark Souls 2}, the tutorial is not really difficult, but some of its elements are harder to find than usually in games, and requires a little effort on the player's part to get all the information they need. Points of interest are scattered around the starting area and it is not always clear where it would be best to go next. We could say that a player of the core audience is the kind that will usually search the areas in the game carefully and be interested and persistent. In other types of games the takeaway most often seems to be that you should start simple and build from there.
\paragraph{2. The tutorial should reflect the pacing of the gameplay and not introduce mechanics before they are accessible in-game.}
In games that have character development and skill unlocks etc. this feels best for an optimal learning experience. This also means that it is not necessarily a good idea to have one tutorial at the beginning of the game, but rather spread it along the game and introduce new concepts as they become available.
\paragraph{3. There can be multiple tutorials for different aspects of the gameplay.}
Multiple tutorials become a better idea when there is a lot to take in with a single concept within a game. \textit{Heroes of the Storm} does this efficiently with a separate tutorial for the core gameplay, and another one for the various map mechanics the different arenas have. It is a thing of its own to learn basic movement, attacks and skills. When the player is more comfortable with these, learning the larger functionality like what options different arenas give becomes more relevant.
\paragraph{4. Controls should be taught interactively when possible.}
When introducing a concept, making the player try it right away is a good example of learning by doing. It is another thing to acquire knowledge by watching or reading, but being able to apply it right after seems to work well in these video games as it does in other things.
\paragraph{5. Learning through example is a possibility.}
\textit{Quake Live} shows the player video demonstrations with the tutorial. In some games, actions require more than just pressing a button, and \textit{Quake Live's} strafe jumping is a good example of this. It requires precision and combining different controls with a very exact timing pattern. In these kinds of settings having a video demonstration helps understand the concept a lot better, rather than just showing which buttons to press.
\paragraph{6. Using a training opponent before the actual game is a possibility.}
This is a method to guide the player in gently, before engaging actual human opponents in multiplayer games. A computer opponent can adjust to the player's skill level and introduce the game dynamics in a more understable way, rather than having the player just jump in with human players who can be many times better than the new player, making the experience frustrating and hard to grasp.
\paragraph{7. Sometimes it is good to limit the possible controls to the ones being currently taught and pace the tutorial accordingly.}
When there is a number of different controls in a game, it can confuse the player if they accidentally do something they are not supposed to do yet in the tutorial. This is why some games choose to limit the buttons and keys that have actions to only the ones that are currently being presented. This way, a more natural flow of concepts can be achieved, and the player knows exactly the thing they are supposed to do and what causes it.
\paragraph{8. Teach applications and combinations of the basic controls that go beyond the basic control scheme.}
A game's controls can be more than the sum of their parts. This is very popular among fighting games. Basic attacks are one thing, but chaining them together to create combinations of actions that are even more powerful is an important aspect to learn. \textit{Guilty Gear Xrd Revelator} achieves this nicely with an interactive tutorial showcasing one set of possible combinations. Since there are many different possibilities of combinations it is not really possible to show them all in a tutorial, and one fun aspect of a game can be to find new ones yourself. It is rather the planting of the idea that this type of thing is a possibility.
\paragraph{9. Using a separate video to describe the usual game elements and dynamics is a possibility.}
It can be difficult to be aware of the type of dynamics, for example, a typical multiplayer match in a game contains. Having a video to go over the basic concepts and gameflow can help understand the bigger picture in question, and prepare the player for the possible options within a match.
\paragraph{10. A short tutorial for a complex game can still be good.}
\textit{Dead Cells} is a complex game but at least so far, being in early access, it does not boast an extensive tutorial. This does not really seem to matter. The game's controls are simple and can be presented quickly. It is rather the large number of weapons and equipment available that create complexity, also making it easy to learn but hard to master. 
\paragraph{11. Use different visual and audible ways to present information and color code keywords.}
Sight and sound seem to enhance the learning experience and help differentiate the core concepts from other content, as the case is with \textit{The Banner Saga}. Showing more important words with different colors and using blinking effects to help notice which parts of the interface should be pressed next make it easier to understand the whole and can speed up the tutorial process too.
\paragraph{12. Have an option to turn off in-game tutorials.}
When a player gets more experienced with a game, this type of information can sometimes only get in the way of more important information of what is actually happening within the game. Being able to turn such things off can enhance the usability of a game.
\paragraph{13. Real-world scenarios can be copied to mimic training sessions in applicable genres.} In our review here this is mostly true for tactical shooters, but could also be applied to other types of games that aim to be more realistic representations of real-world scenarios, like driving games or flight simulators. Here, \textit{Insurgency} and \textit{SWAT 4} offer versions of actual police training courses that also make the game experience more immersive. If the game designer can think of such an example, it is worth looking into since someone can basically have already made the core of the tutorial for them.
\paragraph{14. Consider not showing all possible information depending on the setting, lack of knowledge can serve as a game dynamic.}
\textit{This War of Mine} tries to show what it can actually be like to live and survive in a warzone. Depending on the goals of the game, it can be beneficial to use such elements from real life. The player does not necessarily know if everything they find is usable or beneficial in some way, which helps create an oppressive atmosphere combined with restrictions in, for example, time and inventory. If everything was taught in a tutorial it could make the game too easy and miss its point.
\paragraph{15. Completing a whole mission step by step in an applicable genre can be helpful.}
This is related to showing a video of actual gameplay like in \textit{Quake Champions}, but for such a fast paced multiplayer game it is probably a better option than what the developers have done with \textit{Hitman}. Being a more slowly paced action \& stealth game brings the possibility of playing through a whole mission yourself with the help of a tutorial. It takes you through an example run of a mission that can be completed in a number of different ways, showing you one possible playthrough. This gives a good overview of the types of things that are possible, and counters the problem of not knowing what to do text, which can be frustrating when first playing a game. Being able to interact with the game from the start and learning by doing makes it a good choice for this type of game.
\paragraph{16. Use context-sensitive information throughout the game.}
In some games, not all actions are possible at all times. They can be related to, for example, the positioning of the character in the game. It becomes helpful to have a cue on when a certain action is possible. An example of this is the character being able to take cover or vault over an object. Removing the uncertainty on what controls are available at a given moment can also help remove frustration from the player.
\paragraph{17. Do not underestimate the importance of a tutorial even in a simple game.}
Not knowing how to interact with the game---especially when the context of controls is not the typical industry standard mappings in a popular genre---can make it extremely difficult to proceed, even in a seemingly simple game. \textit{Minecraft: Pocket Edition} offers basically no information when the game starts, and the player is left tapping the phone screen trying to figure out what gestures and actions do what. However simple the controls are, they should always be presented in some way. It is a huge assumption to make that any player can figure things out on their own, as they will eventually give up and forget about the game.
\paragraph{18. Present simple things in a clear and noticeable manner.}
\textit{Monument Valley} is a great example of a simple game that presents things in a clear and understandable manner. Even with just two controls presented it can be noticed how much more fluid the whole experience becomes. Another key point here is that even if there are few controls, it is important that the player still knows exactly how many there are. It removes the frustration of having to think about if there is something else they could do or if they are missing something, and they can better focus on the task at hand.
\paragraph{19. Consider a separate practise or free mode that can be used at the player's own pace.}
Being able to practise the more complicated mechanics or gameplay in a separate setting brings the repetition that is important in acquiring skills. This might not be necessary for simple actions like jumping, but strafe jumping in \textit{Quake Live} or shooting curve balls in \textit{New Star Soccer} are things that can greatly be improved outside of the core gameplay as well. If the only way to practise would be in actual matches the experience could become frustrating if the player's intention was to specifically practise certain elements in the game.
\paragraph{20. Realism is not an excuse to not have additional visual elements in a tutorial setting.} A game can aim to be realistic, but certain aids should not be forgotten at the expense of realism when building a tutorial, like with \textit{Insurgency} or \textit{SWAT 4}. Using arrows and visual aids in the environment to help the player understand different given objects could even be seen as a form of augmented reality within a virtual environment.
\paragraph{21. The tutorial can also be about what the character can or should do, not just how.}
\textit{Hammerwatch} does a thing where the player is shown different possible options on the map tiles as they move around the game world. These include things like what kind of passages are usable, not just what buttons the player needs to press to perform actions. If a wider understanding of the possibilities in a game world is thought to be in order, the environment can help deliver that information.
\paragraph{22. The tutorial should not be made unnecessarily difficult.}
Even in a game like \textit{Dark Souls 2} where the tutorial is a little more ambiguous it is still not overly difficult. It can be argued that a tutorial with the necessary basic mechanics of the game should prepare the player for the game, no matter how difficult the actual game would then be. The tutorial makes itself practically unjustified if it challenges the player in a way that they cannot be expected to expect.
\paragraph{23. The goals of the tutorial should be clear at all times.}
No matter how the tutorial is presented, it should never prevent progress in the game in a way that the player does not understand. This is why it is a good idea to have the status of the system visible, as is suggested already in the traditional Nielsen's heuristics. Knowing what the player has to do and how to achieve it should never leave the user interface, as that would make progress a lot harder which is not desirable in a tutorial. This starts from simple things, like knowing that I have to jump, and I need to press this button to achieve it.
\paragraph{24. The tutorial should begin with the most common interaction in the game.}
What you do most should be learned first. There is no use to know difficult combo attacks in a fighting game if the player does not yet have a grasp on the basic controls, or whatever interaction in a game is done most often. 
\paragraph{25. The player should be able to skip the tutorial but not by accident.}
Having to play a tutorial again without wanting or needing to leads to a frustrating experience. This why it is important to be able to skip it, but in a way that does not become unintentional. Having an additional dialogue, for example, to confirm that the player actually wants to skip the tutorial after pressing a certain button is an easy way to make sure there are no unwanted results.
\paragraph{26. The tutorial should give feedback of the player's required actions with visual and audible elements, and controller vibration if possible.}
This can help enhance the learning process and memorization of controls and elements of the game. When there are more senses in use the experience can be better rooted and also makes the feeling of progress more enjoyable.
\paragraph{27. The player should be able to save the game during the tutorial.}
Some games have long tutorials in the beginning of the game that cover a lot of different topics. It can happen that the game crashes or there is some other reason the player cannot complete the tutorial in one sitting, as happened a number of times with \textit{Insurgency}. Having to play it again from the beginning can be a very frustrating experience. This is why it can be a good idea to implement save game functionality starting already from the tutorial. This is not a big problem for games that have implicit tutorials, as it is usually the norm that the game can be saved from the very beginning, but explicit tutorials do not always work this way. They are a section separate from the main game, but could still benefit from the player being able to complete it at their own pace.

\section{How to use the heuristics}

As said, these heuristics are not to be treated as a list of elements that all need to apply to a single tutorial. Games come in so many forms that it cannot, nevertheless, be expected, and is not intended as a specifier for a good tutorial either. What matters is that when specific heuristics can be taken into account, it might generally be a good idea to do so, and not think that any of them are overlapping with each other. There are different types of learners (e.g.\ visual, aural and kinetic), and different styles of presentation are better for others, but often a combination of styles is even better. To further expand on this, there are also different knowledge acquisition styles: explorative and modeling. \textit{Explorative acquisition} has to do with trying everything and taking risks. Pushing all the buttons and pulling all the levers, seeing what happens. \textit{Modeling acquisition} is wanting to know how things work before trying them out. For example, knowing how the interface works before moving on to the rest of the game, and being able to try things out multiple times and making sure they understand it before proceeding. These two styles can also be combined by making sections of the tutorial skippable, so the explorative player does not get frustrated with having to follow on a set path before moving on to try things out on their own. \cite{Ray2010} Even the need for a tutorial altogether can be challenged, as has been done with Minecraft: it can drive the social interactions of a multiplayer game in a positive way, when players try to find out things collectively. \cite{Wawro2015} The Minecraft we studied, however, was the mobile version, which has a different control scheme, making it a different kind of experience to approach from the desktop version of the game.

27 is a relatively large number of heuristics here, but variations in genres and mechanics can already limit that number considerably. Since there can also be a link between game genre preferences and personality types, it might not also be a good idea to artificially force cross-genre tutorial conventions, at least if player retention is a goal \cite{Peever2012}. This means that unfamiliar genres might drive some people away, which goes back to our heuristic number one, stating that the tutorial should reflect the intended gameplay experience. One example could be the adding of a timer to a turn-based game's tutorial to force the player into quick actions, whereas the regular gameplay would still normally be relaxed and the player could decide when to end his turn. What matters is understanding what is necessary for the game and what it is trying to convey, and even what the target audience is like, if there is such a thing being considered. One distinction that can be made in this context is thinking about the so called casual and hardcore gamers. Casual gamers are ones who only play games a few times a year and in shorter sessions. Hardcore gamers, on the other hand, play games more often and for longer periods at a time. It can be difficult to design a tutorial that would please both groups, so an adaptive approach has been suggested, where the main point is giving the option to skip the tutorial. \cite{Moirn2016} This, then, goes back to our heuristic number 25: the player should be able to skip the tutorial but not by accident. Even if a game's target audience is more hardcore than casual, it can still help to structure the tutorial in an intuitive manner: teaching mechanics in order of complexity, and not introducing combinations of controls before showing them individually, even showing an example of what should be done before making the player try it. The controls alone can be difficult already and require dexterity. Making the player figure out the correct way to use the controls in combination in an unfamiliar context can be frustrating, so an example could help. \cite{Bycer2017} Going through our list of heuristics a developer is hopefully able to easily identify the heuristics applicable to the game and genre in question, and also tailor it to better suit their target audience.

\begin{displayquote}{\textit{``Was that a lot to remember?
Good, because a lot of servants are about to attack you.
Defeat them all.''}} --Jack-O' Valentine, \textit{Guilty Gear Xrd Revelator} tutorial
\end{displayquote}











