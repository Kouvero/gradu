\chapter{Analyzing expert review results}

How to combine our experience with the games in the previous chapter with the heuristics we looked at earlier? Since there are games from various genres it is justified to have a somewhat abstract take on how they are or should be presented. It also depends a lot on the context, since principles used in a good fighting game tutorial do not necessary apply to a game of another genre that well. This means that when thinking about how to build a good tutorial, you only need some tools from the toolbox, but a versatile toolbox is still a good thing to have. If there is no hammer, it does not mean that nails are not a good thing. This is also partly the case with Minecraft: Pocket Edition. If there was a tutorial, the game might turn out to be easy and fun to play, but the lack of one demonstrates that good and usable things are not necessarily intuitive at first \cite{Raskin1994}. This is the reason it is important to teach the player, no matter how simple the controls would be. One related example is text and programming editors such as Vim. They are not intuitive to use and require a long process to learn and modify to your own purposes, but eventually large boosts in productivity may be experienced \cite{Robbins2008}. However, games are in most cases a voluntary pursuit for the player, not a tool, and are works of imagination and art to be enjoyed mostly as entertainment in various genres, so making a one size fits for all collection of heuristics is rather impossible. Similarly, one could also say that the game should not be a chore to learn, no matter how fun it might become after hours of learning. It is important to have tools to analyze tutorials and games, but games that in itself feel like a chore to the player rarely succeed. In other words, productivity software shoud aim to make things easy and efficient, whereas games should be fun and challenging enough \cite{Pagulayan2003}. 

One central thing emerges from the concept of a tutorial: it does not generally matter if you have a separate tutorial, or a sort of in-game help and instruction that guides the player in to the game. This can even happen for a very long period of time, as a game gradually introduces new concepts as the characters develop and e.g. gain new abilities for hours into the game, up until the very last levels even. It is not always just "learn everything and start playing". The game design choice of having mechanics come up later in the game means that it is not necessarily a good idea to teach all those mechanics beforehand in a tutorial at the very beginning of the game. Having small \textit{separate} tutorials along the game could be distracting in the long run, so it is better to embed them within the actual game as new properties are unlocked. The tutorial, then, becomes a little ambiguous as a concept. A game might instruct the player really well but not have a separate tutorial to do that, or the game might have a separate tutorial and also instruct the player really well in that as well. It is safe to say that when we are talking about a tutorial, we can talk about the way the game attempts to transfer the required knowledge to the player, as we defined tutorials in an earlier chapter. A common nominator is that the tutorial should begin with things of lower complexity and then move on to higher level concepts. This is also called \textit{priority learning} \cite{Bycer2016}. With these results in mind, we will go through the game usability heuristics found in the literature and discussed before, and attempt to enhance our view on how video game tutorials can be analyzed regarding their usability based on the games we discussed.

\section{Tutorials and existing heuristics}
\paragraph{1. Easy to learn, hard to master.} How can we say that a tutorial should be easy to learn, but hard to master? In the case of games like Dark Souls 2, the presence of a tutorial might not be completely evident, but it is still not a very difficult tutorial. If anything, if there is a tutorial, the whole point of it is to transfer information. This means that if there is one, it should do its job well, or not exist at all, like with This War of Mine. The lack of a tutorial in Minecraft: Pocket Edition is different, because there is nothing on the screen that tells us that there are possible actions, unlike in This War of Mine. Using this heuristic for a tutorial, we can just say that the game---or rather the tutorial---should be \textit{easy to learn}.

\paragraph{2. The goals are clear. } This heuristic applies to tutorials as well. They should be clear about what they aim to teach and what the player's next action should be if applicable.

\paragraph{3. The skills needed to attain goals are taught early enough to play or use later, or right before the new skill is needed.} When the game design is such that all of the possible actions are not immediately accessible, like in Hori and the Blind Forest, it is a good idea give that information to the player only right before the new mechanic becomes available to the player. 

\paragraph{4. The first ten minutes of play and player actions are painfully obvious and should result in immediate and positive feedback for all types of players. } This seems like something that is present in all of the games except Minecraft: Pocket Edition, which shows that this lack of information can have a very detrimental effect on learning the game even with a seemingly simple game.

\paragraph{5. Player does not need to read the manual or documentation to play.} What is documentation? If there is a tutorial that has text in it, even if presented in small parts, does it not essentially mean that the player is reading the documentation? Accessing a separate document or file, however, is not necessarily desirable. Also, there are complicated genres like flight simulators where a huge documentation is a necessity. Having all that information presented in-game in a fluid manner would still be a good thing, so this heuristic is a good goal in that sense, but not always practical.

\paragraph{6. Player does not need to access the tutorial in order to play.} The ability to skip a tutorial is an important one, just design it so that someone playing the game for the first time does not do accidentally.

\paragraph{7. Status score indicators are seamless, obvious, available and do not interfere with game play.}
This becomes meaningful in e.g. sequenced tutorials where the next action is not available until the current one has been completed, so it is important to communicate that necessity properly.

\paragraph{8. The game provides feedback and reacts in a consistent, immediate, challenging and exciting way to the player's actions.} Like in the previous point, making clear that the player just did the correct required action is important. The Guilty Gear Xrd Revelator tutorial is a great example of this.

\paragraph{9. Provide appropriate audio/visual/visceral feedback (music, sound effects, controller vibration.} This, again, adds to the previous point in question. It should just be considered that there are multiple ways to communicate a success simultaneously, i.e. not just in text but also in color, sound and e.g. controller vibration.

\paragraph{10. The player is given controls that are basic enough to be learned quickly, yet expendable for advanced options for advanced players.} This is very true with a game like Guilty Gear Xrd Revelator. The fact that even the tutorial in the game goes in that depth to demonstrate advanced combinations extended from the basic controls makes this an important heuristic to have.

\paragraph{11. Player interruption is supported, so that players can easily turn the game on and off and be able to save the game in different states.} Insurgency has a moderately long tutorial section, and it had a tendency to crash a number times while playing it. This meant that the tutorial had to be started again from the beginning which was frustrating. Adding a save game option to be used from a separate tutorial context is thus a good idea.

\paragraph{12. Upon turning on the game, the player has enough information to begin play.} This is self-evident in a way. If there is a separate tutorial, it is a good idea to have it accessible from the same menu as the main game, so that the option is clear.

\paragraph{13. Players should be given context-sensitive help while playing so that they are not stuck and need to rely on a manual for help.} 
In a tutorial it should always be clear what the next expected player action is.

\paragraph{14. All levels of players are able to play and get involved quickly and easily with tutorials and/or progressive or adjustable difficulty levels. }
Here, this essentially means that the tutorial should be easily accessible if there is a separate one, or that the in-game tutorial should be good enough according to the other heuristics. In general, it should not take too long for player to be able to actually do something in the game, which is mostly likely some part of the tutorial.
 
\paragraph{15. Get the player involved quickly and easily.}
This is practically same as the previous one. With the core idea being the tutorial, it does not need to be separately mentioned in the later heuristics. Also, as the heuristics regard tutorials, it can be assumed that the tutorial is already in progress, so how quickly the player gets to the tutorial is not in essence the problem. It does become a problem when there is a lack of instruction or a tutorial altogether, like in Minecraft: Pocket Edition. But since we are evaluating the usability of a tutorial, we can assume that there already is a tutorial we are evaluating.

\paragraph{16. The game should give hints,  but not too many.} For progress in the actual game this is likely a good idea, but a tutorial should not give hints. It should be as clear as possible, not "maybe you need to press A to jump, maybe B, who knows...". This might still be enjoyable if humor was the intent, and rules are meant to be broken anyway.

\paragraph{17. Allow users to skip non-playable and frequently repeated content.} This is very important for a tutorial too, just so that it is not done accidentally.

\paragraph{18. Provide instructions, training and help.} These have all been mentioned earlier, and are in a way the essence of tutorials. Training, like it is in New Star Soccer or The Banner Saga, is a great way to enhance the gaming experience in addition to some other basic information that would be considered a tutorial. To be able to practise on your own, not in a guided setting and not having to worry about the outcome too much, can be beneficial.

\section{Heuristics for tutorials}
Now that we have distilled our experience with the games in question through the heuristics used with general game design and usability found from the literature, it is time to think about heuristics for video game tutorials specifically. One difficult thing with this is the wording of heuristics. As discussed earlier, not all heuristics need apply in order for the tutorial to be efficient. It also depends a lot on the genre of the game for example. There is a tendency with heuristics to have a sort of a "you should do this" type of tone, whereas in practise, they are only things to be considered that might fit the game's needs or not. We need to think about how to present the information so that everything does not come across as a necessity, but rather a point to consider. Not all games necessarily need a training mode for example, so it can be problematic to say "have a training mode". Using the takeaways from the experiences with the games earlier and the general heuristics discussed in the context of tutorials, we can combine them to create a list of heuristics that is more applicable to the usability of video game tutorials specifically. We were able to identify and create a total of 27 heuristics, which are as follows:

\paragraph{1. The tutorial should reflect the intended gameplay experience.} 
\paragraph{2. The tutorial should reflect the pacing of the gameplay and not introduce mechanics before they are accessible in-game.}
\paragraph{3. There can be multiple tutorials for different aspects of the gameplay.}
\paragraph{4. Controls should be taught interactively when possible.}
\paragraph{5. Learning through example is a possibility.}
\paragraph{6. Using a training opponent before the actual game is a possibility.}
\paragraph{7. Sometimes it is good to limit the possible controls to the ones being currently taught and pace the tutorial accordingly.}
\paragraph{8. Teach applications and combinations of the basic controls that go beyond the basic control scheme.}
\paragraph{9. Using a separate video to describe the usual game elements and dynamics is a possibility.}
\paragraph{10. A short tutorial for a complex game can still be good.}
\paragraph{11. Use different visual and audible ways to present information and color code keywords.}
\paragraph{12. Have an option to turn off in-game tutorials.}
\paragraph{13. Real-world scenarios can be copied to mimic training sessions in applicable genres.}
\paragraph{14. Consider not showing all possible information depending on the setting, lack of knowledge can serve as a game dynamic.}
\paragraph{15. Completing a whole mission step by step in an applicable genre can be helpful.}
\paragraph{16. Use context-sensitive information throughout the game. }
\paragraph{17. Do not underestimate the importance of a tutorial even in a simple game.}
\paragraph{18. Present simple things in a clear and noticeable manner.}
\paragraph{19. Consider a separate practise or free mode that can be used at the player's own pace.}
\paragraph{20. Realism is not an excuse to not have additional visual elements in a tutorial setting.}
\paragraph{21. The tutorial can also be about what the character can or should do, not just how.}
\paragraph{22. The tutorial should not be made unnecessarily difficult.}
\paragraph{23. The goals of the tutorial should be clear at all times.}
\paragraph{24. The tutorial should begin with the most common interaction in the game.}
\paragraph{25. The player should be able to skip the tutorial but not by accident.}
\paragraph{26. The tutorial should give feedback of the player's required actions with visual and audible elements, and controller vibration if possible.}
\paragraph{27. The player should be able to save the game during the tutorial.}

\section{How to use the heuristics}

As said, these heuristics are not to be treated as a list of elements that all need to apply to a single tutorial. If such a tutorial exists where this is possible, good. Games come in so many forms that it can not nevertheless be expected, and is not intended as a specifier for a good tutorial. What matters is that when specific heuristics can be taken into account, it might generally be a good idea to do so. 27 is a relatively large number of heuristics here, but variations in genres and mechanics can already limit that number considerably. Since there can also be a link between game genre preferences and personality types, it might also not be a good idea to artifically force cross-genre tutorial conventions, at least if player retention is a goal \cite{Peever2012}. This means that unfamiliar genres might drive some people away, which goes back to our heuristic number one, that the tutorial should reflect the intended gameplay experience. What matters is understanding what is necessary for the game and what it is trying to convey, and even what the target audience is like, if there is such a thing being considered. 

\begin{displayquote}{\textit{"Was that a lot to remember?
Good, because a lot of servants are about to attack you.
Defeat them all."}} --Jack-O' Valentine, Guilty Gear Xrd Revelator tutorial
\end{displayquote}











