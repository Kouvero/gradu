\chapter{Analyzing expert review results}

How to combine our experience with the games in the previous chapter with the heuristics we looked at earlier? Since there are games from various genres it is justified to have a somewhat abstract take on how they are or should be presented. One central thing emerges from the concept of a tutorial: it does not generally matter if you have a separate tutorial, or a sort of in-game help and instruction that guides the player in to the game. This can even happen for a very long period of time, as a game gradually introduces new concepts as the characters develop for hours in. It is not always just "learn everything and start playing". The game design choice of having mechanics come up later in the game means that it is not necessarily a good idea to teach all those mechanics beforehand in a tutorial at the very beginning of the game. Having small separate tutorials along the game could be distracting in the long run, so it is better to embed them within the actual game as new properties are unlocked. The tutorial, then, becomes a little ambiguous as a concept. A game might instruct the player really well but not have a separate tutorial to do that, or the game might have a separate tutorial and also instruct the player really well in that as well. It is safe to say that when we are talking about a tutorial, we can talk about the way the game attempts to transfer the required knowledge to the player, as we defined tutorials in an earlier chapter. 



