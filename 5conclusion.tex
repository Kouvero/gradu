\chapter{Conclusion}
Starting with the discussion on what usability is, we have come to a list of heuristics that can help us to evaluate the usability of video game tutorials. Along the way a thought arises that usability seems generally more universal as a concept than the usability of video game tutorials. This means that the usability of a tutorial cannot necessarily be generalized in the same way or as easily as the general principles for the usability of an interface may suggest, which is what a lot of traditional usability literature and research seems to be concerned with. We did not find heuristics dedicated to tutorials alone. There is much more going on in video game usability than just the interface. Games come in different forms and genres, and usability therein---especially with tutorials---has more layers than just the interface of the game. The information is more dynamic and in a state of change, as new concepts and mechanics are introduced on the screen, the guides usually vanishing soon after. Tutorials generally aim to convey information about how the game is actually played, and designing an interface that would alone fill that requirement is rarely, if ever, satisfactory. As the way the game can be played becomes more familiar, it would be a waste of space to keep the tutorial information on the screen all the time, which means that it becomes more important when, where and how to show the necessary information and tips. 

We can also see a divide between productivity software and video games, which are another type of software. A game is usually a contained system that we engage with voluntarily, so the requirement for good usability and guidance becomes even more important, because the first hours of the game are the most important time to keep the player playing the game. Productivity software involving, for example, word processors and database management are necessary tools for certain fields of work, and can be connected to a larger whole. This can mean modifying a database which causes boxes to be moved in a real-world warehouse and shipped to another country. It cannot always be predicted what type of work will the end users be doing. Productivity software is also usually more complex than games and come with manuals and extensive documentation, which is not desirable with games, as heuristics in the previous studies suggest. However, a game-like approach has been tried with productivity software tutorials too, with good results, using principles of gamification \cite{Li2012}. The methods there also shared a resemblance with the heuristics in this thesis, e.g. in that they where interactive and used contex-sensitive immediate feedback on user actions. 

There is still something to be learned from standards like the ISO/IEC 9126-1 that we can apply to game design as well. The standard states, among other things, that the software's quality in use should fulfill certain goals, such that the product enables users to achieve specified goals (in this case learning the game) with effectiveness, productivity, safety and satisfaction in specified contexts of use. Compared to productivity software, video games, however, know themselves inside out, and tutorials can be designed and tailored to the whole experience and the possibilities it offers within a single game. This is not always the case in practise, and heuristics for tutorial design can be a welcome addition to the development process of the game. But is it enough that tutorials are efficient? Do they also have to be fun? Is fun the responsibility of the core game, not the tutorial, or both? If the core game is fun to play, according to our first heuristic the tutorial should reflect that, so a fun game's tutorial should hopefully then increase player retention as well. Whether a challenging game's tutorial should also be challenging is a more difficult question. Still, player retention at the beginning of the game is not all about the tutorials, since some tutorials last for a very long time and are not a separate section at the very beginning of the game. Tutorials can be coming up many hours into the game (explicit vs. implicit tutorials), and \textit{when} the helpful information should be presented then becomes a more important question, and also more specifially, \textit{what} information. It is not a good idea to start from the most complicated aspects of the game.  In general, games can still have many options for which actions to take at any given time, which is why it is not necessarily a good idea to teach everything at once, but rather do it in steps and right before the player takes a certain action.

These usability heuristics are essentially rules of thumb derived from testing the heuristics in the literature combined with the author's personal experience with a number of chosen games from different genres, as listed in Chapter 4. From this combination we have formulated a list of heuristics specifically for tutorials. The heuristics are not meant to be applied all at one to a single game. Theign of designer needs to use their own judgement whether a given heuristic is applicable to their game or not, or whether any of them are. The list of games we have looked at here contains titles from a number of different genres. This is because the idea has been to try and recognize improvements on a larger scale, not within a single genre. No two games are exactly alike, so having a set of heuristics that can be used with designing a game of any genre or style makes the result applicable on a wider scale. Also, any single heuristic is not bound to a single genre. Their usefulness can be realized in any context by going through the list and seeing if it contains an applicable idea for the game or tutorial that is being developed. It can be useful because different games can borrow things from different genres and mix things up, so the divide between game styles does not have to be set in stone or defined in an extremely accurate way. 

Judging from the success of certain recent years' hit games such as the \textit{Dark Souls} series---one of which we analyzed here---it can also be argued that if you set out to make a game based on certain recognized industry standards and principles and an enjoyable experience in mind, you might have already taken your first wrong turn. Not all people like difficult and punishing games with death of the player as an integral mechanic, and where things are not explained to you in a hand-holding manner, but such games can still become recognized classics. Different target groups can want two opposite things, so it is difficult to generalize, and in the game development process that is a whole another thing to take into account, that is, who are you making your game for, if that even is a question. So, if you are brave and creative enough, heuristics such as these can also give you ideas on what to avoid, not which rules to follow.
