\chapter{Conclusion}
Starting with discussing what usability is, we have come to a list of heuristics that can help us evaluate the usability of video game tutorials. Along the way a thought arises that usability seems generally more universal as a concept than the usability of video game tutorials, meaning that the usability of a tutorial can not necessarily be generalized in the same way as easily as the general principles for the usability of an interface may suggest, which is what a lot of traditional usability literature and research seems to be concerned with. There is a lot more going on in video game usability than just the interface. Games come in different forms and genres, and usability therein---especially with tutorials---has a lot more layers than just the interface of the game. The information is more dynamic and in a state of change. Tutorials generally aim to convey information about how the game is actually played, and designing an interface that would alone fill that requirement is rarely, if ever, satisfactory. We can also see a divide between productivity software and video games, which are another type of software. A game is usually a contained system that we engage with voluntarily. Productivity software involving e.g. word processors and database management are necessary tools for certain fields of work, and can be connected to a larger whole. This means e.g. modifying a database which causes boxes to be moved in a real world warehouse and shipped to another country. It can not always be predicted what type of work will the end users be doing, which is perhaps a reason tutorials are not a thing with productivity software. They are also usually more complex than games and come with manuals and extensive documentation, which is not desirable with games, as heuristics in previous studies suggest. Video games, however, know themselves inside out, and tutorials can be designed and tailored to the whole experience and the possibilities it offers within a single game. This is not always the case in practise, and heuristics for tutorial design can be a welcome addition to the development process of the game. These usability heuristics are essentially rules of thumb derived from testing in the literature combined with the author's personal experience with a number of chosen games from different genres, as listed in chapter four. The heuristics are not meant to be applied all at one to a single game. The designer needs to use their own judgement whether a given heuristic is applicable to their game or not, or whether any of them are. 

The list of games we have looked at here contains titles from a number of different genres. This is because the idea has been to try and recognize improvements on a larger scale, not within a single genre. No two games are generally alike, so having a set of heuristics that can be used with designing a game of any genre or style makes the result applicable on a wider scale. Also, any single heuristic is not bound to a single genre. Their usefulness can be realized in any context by going through the list and seeing if it contains an applicable idea for the game or tutorial that is being developed. It can be useful because different games can borrow things from different genres and mix things up, so the divide between game styles does not have to be set in stone or defined in an extremely accurate way. 

Judging from the success of certain recent years' hit games such as the Dark Souls series---one of which we analyzed here---it can also be argued that if you set out to make a game based on certain recognized industry standards and principles and an enjoyable experience in mind, you might have already taken your first wrong turn. Not all people like difficult and punishing games with death of the player as an integral mechanic, and where things are not explained to you in a hand-holding manner, but such games can still become recognized classics. Different target groups can want two opposite things, so it's difficult to generalize, and in the game development process that is a whole another thing to take into account, i.e. who are you making your game for, if that even is a question.
