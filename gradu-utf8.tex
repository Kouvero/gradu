\documentclass[a4paper,12pt,twoside]{article} % kaksipuolinen
\usepackage[finnish]{babel}            %suomenkielinen tavutus ja sanasto
\usepackage[T1]{fontenc}               %valitaan ääkkösfonttikoodaus
\usepackage[utf8]{inputenc}        	% skandit utf-8 koodauksella
\usepackage{graphicx}           %kuvat
%\usepackage[dvips]{graphicx}           %ps-kuvat

\setlength{\oddsidemargin}{1.85cm} %kaksipuoliset marginaalit
\setlength{\evensidemargin}{0.35cm} %kaksipuoliset marginaalit

\begin{document}

\pagestyle{empty}  %ei sivunumeroa sivun alareunaan

\begin{center}
\includegraphics[width=4cm]{soihtu.eps} %talleta kuva linkistä omaan hakemistoosi
\end{center}

\vspace{3.0cm}
\begin{center}\large
GRADUN OTSIKKO
\end{center}

\vspace{0.5cm}
\begin{center}
Ewert Kupiainen
\end{center}

\vspace{0.5cm}
\begin{center}
Pro gradu -tutkielma\\
Huhtikuu 2013
\end{center}

\vspace{5.0cm}
\begin{center}
MATEMATIIKAN JA TILASTOTIETEEN LAITOS\\
TURUN YLIOPISTO\\
\end{center}

\cleardoublepage

%säädetään 2-puolisen sivun marginaalit vaatimuksia vastaaviksi
\addtolength{\oddsidemargin}{-1.1cm}
\addtolength{\voffset}{-2.3cm}

\begin{minipage}{15cm}

\noindent
TURUN YLIOPISTO\\
Matematiikan ja tilastotieteen laitos\\
\\
KUPIAINEN, EWERT: Tiivistelmäsivun laadinta LaTeX-ladontaohjelmalla\\
Pro gradu -tutkielma, 39 s., 12 liites.\\
Matematiikka\\
Huhtikuu 2013\\
\rule{\textwidth}{.2mm}\\
\\
Tiivistelmäsivun tekeminen LaTeX-ladontajärjestelmällä on helppoa. Kopioidaan esimerkiksi www-sivulta tiivistelmäsivun LaTeX-koodi, jossa on valmiiksi oikeanlaiset marginaalit.

\vspace{4mm}\noindent Sitten vain kirjoitellaan erittäin fiksua tekstiä tiivistelmäsivu täyteen. Muistetaan laittaa \noindent-komento kappaleen
alkuun, niin LaTeX ei sisennä ensimmäistä riviä.

\vspace{4mm}\noindent Tuo $\backslash$vspace-komento taasen jättää sopivan välin kappaleiden väliin - 4mm näyttää aika hyvältä.

\vspace{4mm}\noindent Muistetaan lopettaa ajoissa ympäripyöreän liirumlaarumin kirjoittelu.

\vspace{4mm}\noindent Asiasanat: tiivistelmäsivu, Pro gradu -tutkielma, LaTeX-ladontajärjestelmä.

\end{minipage}

\cleardoublepage

\pagestyle{plain} 
\setcounter{page}{1}

%palautetaan 2-puolisen sivun marginaalit oletusasetuksiin
\addtolength{\oddsidemargin}{1.1cm}
\addtolength{\voffset}{2.3cm}

Tästä alkaa työ.

\end{document}
