\chapter{Chosen games and their tutorials}

In this chapter, we will go through a list of video games and their tutorials. The tutorials are not identical in each, meaning that one game might have a dedicated tutorial section apart from the main game, where another game might have an integrated tutorial with the actual gameplay and gradually guiding the player in the game from there. Generally there are not usually separate tutorials, but rather the information is embedded in the beginning of the actual game. Perhaps this is to make sure that anyone playing the game gets the information they need, rather than having to find it from a tutorial separate from the main game, but it also depends on the game type. In a fighting game like Guilty Gear Xrd Revelator, the separate tutorial can be justified because there is actually only one game mode, which is two characters fighting. Adding a tutorial to a regular match would be difficult because it would possibly have to be constantly paused and would not feel like the actual gameplay anyway. On the other hand, a more complex tactical  shooter like SWAT 4 has a separate tutorial section that shows you all the important mechanics in the game where you need to command your team and have a lot of controls and options for engagement, which also reflects the realistic setting in a more immersive way.
\section{Initial impressions and takeaways}

The goal here is to establish a general view on these games and how their tutorials work, if there is one. Observations will be made about the user interface, mechanics, presentation of information, the general flow and feel of the tutorial and anything else that comes to mind. As games are often very different from one another, no general style of approach will be defined at this point. Rather, we want to see how the tutorials operate and if there is anything we could think about as a heuristic method that could be applied to tutorial design in general. These will be laid out in the takeaways section of each game. All of the games are PC versions found on the Steam platform unless otherwise noted, the secondary platform being Android. Name of the developer and release year are also listed.

\subsection{Dark Souls 2 (FromSoftware, 2014)}
\paragraph{Genre: Action / RPG \\ Separate tutorial: No \\}
The player can read stone tablets he comes across in the first area of the game, which show the controls separately, one or two basic controls in each tablet. However, they are not presented as something that the player will without a doubt come across. The tablets are scattered in the caves around the area, and a careless player could just run through the area without consulting them. When a tablet is in front of the player character, a prompt comes up that says to push a button to interact, and when the player does text is displayed about how to perform the action written on the tablet. The tablets contain a lot of basic controls and introduces enemies the player can try the controls out on. Which button to press to attack is introduced first, then how to lock a target, how to dash and so on. Other controls presented include how to perform critical hits, rolling, switching weapons, backstepping, how to use an item, how to move the camera around the character, dual-wielding and a dashing jump for distance. Enemies and chasms to jump over come about as the player finds more tablets in the caves. Parrying, guard breaking and how to perform a plunging attack by dropping on an enemy are introduced last. There are basically no basic controls left untold, it is just up to the player to find them as it is in the core of the Dark Souls experience that there is no hand-holding. The player is expected to find out things on his own most of the time, often so by dying and trying again. When death is a game mechanic, it also feels like a design choice to leave the responsibility to the player to explore the first areas of the game as well as they can.
\paragraph{Takeaways:}
The tutorial reflects the intended gameplay experience. It is not supposed to be forced on the player, even in the early stage, since the philosophy of the game is to be unforgiving and requires research on the player's part in general. 

\subsection{Ori and the Blind Forest (Moon Studios, 2015)}
\paragraph{Genre: Platformer \\ Separate tutorial: No \\}
As the player moves for the first time, the game displays information to hold the A button on the gamepad longer to jump higher.
Then to hold down and press A to 'jump down through platforms'.
The game also let's the player know how collectibles work as they are found and how to hold the B button  to save your game (saving is an in-game mechanic, not a separated menu functionality).
Ori keeps introducing new mechanics many hours into the game, and tells you what the controls are in-game, as you unlock new abilities.
There are areas that are not cleary yet accessible until you unlock new abilities (i.e. a double jump) to go there. The game will always prompt you to use the controls accordingly to get there the first time. It's never paused and the tutorial is always integrated to the gameplay.
Ori is a certain type of game where the versatile gameplay is not revealed all at once, but through acquired abilities, escalating the possibilities towards the end of the game and the tutorial follows that same progress as abilities are unlocked.
\paragraph{Takeaways:}
The tutorial reflects the pacing of the gameplay. Teach mechanics in the proper context when they are first needed, not all at once and before they can be even used in the game.

\subsection{Heroes of the Storm (Blizzard Entertainment, 2015)}
\paragraph{Genre: MOBA \\ Separate tutorial: Yes \\}
Heroes of the Storm has two separate tutorials you can choose to play: one concerning character development (called "tutorial") and basic gameplay, other (called "battleground training") concerning the differences of arenas, that have their own special mechanics.
Before the tutorial you have to choose which role to play: assassin, warrior or support.
At the start of the tutorials, you are given a basic overview of the lanes each map has and your objective: to destroy the main structure at the enemy base. You are also told that at least one hero should be present on each line during play. The remainder of the tutorial has the following progress, each bullet point representing what information is given to the player::
\begin{itemize}

\item The overview of the character you chose, i.e. what kind of abilities she has.
\item A "Tip Panel" on the top left of your screen which has the basic controls, and are told to familiarize yourself with them.
\item A short cinematic about how you gain experience (xp) when killing enemy minions and are shown what some basic info on the screen means.
\item information on how your abilities work and are thrown into battle to try them out.
\item As you kill enemies and proceed on the map, you encounter the first turrets, and are told what their mechanics are, which feels very useful (too).
\item The secondary objective(s) each battleground has.
\item You're told to get to the secondary objective, and also how to ride a mount to get there faster.
\item You're told how the secondary objective here, a watch tower, increases your teams visibility around it.
\item You're told to defeat a camp of giants to make them fight on your side for a while down the nearest lane.
\item You're told to go and destroy the enemy base, after which the tutorial ends.
\end{itemize}

In battleground training, you are able choose from the same three hero types (assassin, warrior and support).
When the training starts, you get more information about keyboard shortcuts, such as how pressing tab shows you the score and status card.
There's also a question mark present on the screen at all times from which you can review all the tutorial tips.
Then you get told how to regain mana from fountains or how to teleport back to base to refill your stats.
The tips often pause the game and wait until you've read them.
With the tribute battleground mechanic, you're told very specifically what to do and how it affects the other team if you succeed. This is further enhanced by using specific markers to show locations on the minimap relevant to the tutorial.
When you active the tribute and weaken the enemy team, you are told to attack the enemy and the tutorial ends when their main building is destroyed.
\paragraph{Takeaways:}
Have separate tutorials for different aspects of the gameplay if necessary (e.g. character development and arena-specific mechanics).

\subsection{Psychonauts (Double Fine Productions, 2005)}
\paragraph{Genre: 3D Platformer / Action-adventure \\ Separate tutorial: No \\}
When you start a new game, it eventually throws you into a sequence where you are asked to perform controls, starting with a command to move the right stick towards the right. You are then told about the object you're looking---a collectible---and what it does in the game. After that, you are asked to look up, get told about another collectible there and the sequence ends. After this the game starts.
When you first open the menu, you get a prompt how to move in the menu.
When you go to a tab in the menu for the first time, you get a prompt that tells you what the menu is about, and then click a button the close the prompt.
When you pick up things from the ground for the first time, you get a prompt that tells you what they are and then have to click a button to close the prompt.
You can run around in the first area collecting stuff, but the next area is blocked until you complete an obstacle course that teaches you the rest of the controls. This obstacle course is made in the spirit of a real world obstacle course, such as the training area in SWAT 4, where actual psychonaut candidates are, or would be, trained. First you need to jump over things, then double jump over bigger distances.
The basic training obstacle course is not an easy one and really tests the players abilities with jumping, climbing, punching and collecting things. During the course constant information is given to the player on how to perform these actions and what new collectibles mean.
After the basic training course is completed, the player has been shown everything about the basic controls and gameplay mechanics and they gain access to rest of the areas in the game.
\paragraph{Takeaways:}
Teach basic movement controls interactively, perhaps even before the actual game starts, at least if the controls remain relativle similar for the rest of the game.
Have the first area of the game be a tutorial section that needs to be completed before the rest of the game becomes available.

\subsection{Quake Live (id Software, 2010)}
\paragraph{Genre: FPS \\ Separate tutorial: Yes \\}
Quake Live has a training center where you can optionally get information on how the game is played. There are both videos that show you what you should be doing, and training where you get to try out the different controls yourself. There are three different training sections, called \textit{crash course}, \textit{accelerate} and \textit{elevate}. The crash course shows you the basic controls on how to move and shoot, and is very thorough. You are told how long the tutorial will take and that you can exit at any time by pressing F3. Then you enter an arena with a non-player character (NPC) who gives you detailed information about ammo pickups, weapons, health the map in question and the general playstyle that is considered good (spawn, gear up, fight, restock). Then, a training match with the NPC commences and you fight her in order to complete the first part of the tutorial. The second and third part of the tutorial section teach you how to strafe jump and rocket jump, respectively. The training center apprently wants to make sure you are aware of these mechanics, that are not evident in a general first person shooter (FPS), but doesn't force you to complete the training sections before you can start to play against other people.
\paragraph{Takeaways:}
Think about the possibility of combining video demonstrations with the tutorial.
Use a training match with an NPC character for familiarity before playing against actual people. Have a clear option to exit the tutorial.

\subsection{Rocket League (Psyonix, 2015)}
\paragraph{Genre: Sports / Soccer / Racing \\ Separate tutorial: Yes \\}
Rocket League has a training option in the main menu, from which you can access the game's tutorial, which includes two sections: basic and advanced.
The basic tutorials tells you the basic controls one after the other and requires you to complete different things before it proceeds to the next one. These include tasks like pushing the ball to the goal, jumping at the ball and scoring, making a powerslife turn etc. The tutorial map is a smaller version of the regular game map. All the possible controls are not accessible during the tutorial, only the required ones at any given time.
The same principles apply to the advanced tutorial. You are given more advanced mechanics, but need to proceed with them in a similar way with limited controls to make sure you do things in the required way.
\paragraph{Takeaways:}
When teaching controls, consider only allowing the currently required controls to be available and pace the tutorial accordingly.

\subsection{Guild Gear Xrd Revelator (Arc Systems Works, 2016)}
\paragraph{Genre: Fighting \\ Separate tutorial: Yes \\}
Guilty Gear Xrd Revelator is a 2D fighting game that has an extensive tutorial\footnote{https://www.youtube.com/watch?v=0oWBwcYr1LM}
teaching you many facets of the game. Starting with the movement controls, it
shows you how to move, jump and dash. This is done by making the player pop
balloons with his movements, and obstacles are presented that have to be
jumped over and dashed through quickly enough. All of this makes sure that the correct controls are used in
order to proceed in the tutorial.
After movement training, the tutorial moves on to how to attack in different
ways. The available buttons for attacking are presented on the screen the
whole time, and different targets are given for the player to hit with
specific attacks, with a section dedicated to each attack type (i.e. "attack
all of these targets with the specified attack type"). A timer is also
introduced later. The specified attacks have to be made quickly enough in
order to proceed with the tutorial. The target to hit is also specified with a
symbol and letter specifying the attack to be used, which makes the object
easy to understand at all times. The tutorial also teaches different types of
basic combo attacks (a sequence of single basic attacks), and gives you targets to be beaten with given combos.
This gives the game a very tactile feel from the start, giving practical
applications to the controls instead of just giving you the controls and not
showing how they can be applied and combined. This can feel like the tutorial is a little
minigame in itself, apart from the actual intended gameplay.
The third mechanic the tutorials introduces is blocking, i.e. how to block
your opponents attacks. It also tells for what type of attacks each block
type is effective (for standing, crouching and aerial blocks). NPC's attack
the player, and can only be harmed with a proper counter attack after a
succesful block. A mechanic specific to the Guilty Gear series, a \textit{roman cancel}, is also taught in a similar fashion.
After all this there's a recap of what has been learned (movement, attacks,
blocking and roman cancels), and you need to apply all of the techniques to
beat a number of NPCs combined with an obstacle course forcing the player to
jump and dash in order to complete the section. Finally, there is one last
fight against a more powerful character, and you get tips for how to proceed
with the fight. This means e.g. using a specific type of mid range combo to
build up the meter for using roman cancels for additional combo power.
There's also a separate menu in the game from which you can access match-up
tutorials. These show you how to best deal with specific characters and their
unique abilities.

\paragraph{Takeaways:}
Use a detailed obstacle course to make sure all the required controls are used.
Go beyond the basic controls and teach possible combinations and applications of the controls that will allow the player to start thinking about his own style and variations.

\subsection{Quake Champions (beta), (id Software, 2017)}
\paragraph{Genre: FPS \\ Separate tutorial: Yes \\}
When you first start the game, the first thing you see is an annoucement about
where training videos are found in the menu. At the moment of writing this
they are called beta tutorials, and the tutorials are video only, so the
player needs to try to make use of the information in a separate setting. The
first information given in the introduction video is not about controls, but
rather about the spawning mechanic, weapons and types of power-ups. The first
video of the three is a general introduction with all these concepts. The
second video is about health and armor, and the third one about power-ups.
All collectibles and their functionality is presented. It is assumed that the
player has the required knowledge about the elementary controls. Information about game dynamics is also given, such as how it's important to start gathering power-ups and weapons when the match starts. Emphazising what to do, rather than how to do something, can be a good idea: "spawn, gear up, fight, restock".
\paragraph{Takeaways:}
Consider using a separate video that can be available online to demonstrate the basic principles of the game and how a typical match generally proceeds, not only with mechanics, but dynamics as well.

\subsection{Dead Cells (early access), (Motion Twin, 2017)}
\paragraph{Genre: 2D Platformer / Rogue-lite \\ Separate tutorial: No \\}
Dead cells is a 2D platformer that throws the player straight into the game, and gives context-sensitive tips about how to move the character. Basic (Xbox) gamepad knowledge is assumed with controlling the character with the left analog stick, but a text appears over the character to press 'A' on the controller to jump. Then 'X' for main weapon, to break through a door. Next how to double jump with consecutive A-button presses is shown. Progress is not possible if the instructions are not followed. Then the player gets a secondary weapon and the instructions to press 'Y' to use it to break through another door. The text does not disappear until the button is pressed. Then you get the promp to press 'B' to roll.
\paragraph{Takeaways:}
The tutorial does not have to be any longer than necessary, i.e. a very short tutorial is acceptable. Locking further progress without completing required actions is still an option, however, no matter how simple they are.

\subsection{The Banner Saga (Stoic, 2014)}
\paragraph{Genre: Turn-based strategy / RPG \\ Separate tutorial: No \\}
The game jumps right into a combat tutorial after the first cutscenes. You are told to drag around the screen to see your surroundings, but it is not immediately evident how to track, i.e. by taking the mouse to the edge of the screen or by clicking and dragging. There is time to find out because the tutorial will not proceed until you click a button to continue. Next the initiative chart is explained, which shows the order in which the different characters will act. This also effectively means that The Banner Saga is a turn-based game. You are again asked to click to continue. All the important mechanics are explained and shown one after the other. Different actions and abilities are color-coded the further enhance their meaning. At the start of the next battle, you get the possibility to choose where to place your characters before the battle starts, however this is not mentioned right away, but in a later battle. There is a question mark in the corner which you can click to get tips around the screen about what each item on the screen means. As new game modes or screens are introduced, there is a text pop-up telling what the screen is and how to generally manage it. After the start of the game there is also a training mode accessible, where you can have training battles with your characers.
\paragraph{Takeaways:}
Use visual cues to help locate required actions in the user interface. Keywords and concepts can be highlighted or colorcoded in the tutorial text. Have a help menu accessible from the main interface. A separate mode for practicing the basic game mechanics can be beneficial.

\subsection{Child of Light (Ubisoft Montreal, 2014)}
\paragraph{Genre: 2D Platformer / Turn-based combat \\ Separate tutorial: No \\}
Child of Light is a 2D platformer where you control an additional flying companion character on the screen with the right stick of the controller. The game has a checkbox in the options menu to turn off tutorials, but there is not really a separate tutorial. Basic gamepad experience is assumed, and during the first 90 minutes of gameplay, there are only a few new things presented. These include things like a pop-up that says how to move the companion character with the right stick, and how to press 'A' on the gamepad to fly. There is a certain level of complexity in the controls because the player has to control two characters on the screen simultaneously, which can be a reason why the basic controls feel a little more minimalistic than usual.
\paragraph{Takeaways:}
Have an option to turn off in-game tutorials.

\subsection{SWAT 4 (Irrational Games, 2005)}
\paragraph{Genre: Tactical shooter \\ Separate tutorial: Yes \\}
SWAT 4 has a separate tutorial mode, called \textit{training}, which can be selected from the main menu. It is a shooting and tactical assault practise range that aims to simulate more realistic police training. It shows the player how to operate his firearms, interact with the environment, and most importantly how to command their squad of operatives. Different possibilities for this are quite extensive and are walked through in different parts of the training mode. There is a narrator who follows your progress and talks you through everything, while displaying necessary controls on the screen. Starting from the more simple firearm techniques on to the more complicated ways to command your squad and sniper, it builds in complexity at a steady but manageable rate.
\paragraph{Takeaways:}
In a more realistics setting consider copying a real world training scenario if there is one, or come up with your own as was done in Psychonauts.

\subsection{This War of Mine (11 bit studios, 2014)}
\paragraph{Genre: 2D / Point \& Click \\ Separate tutorial: No \\}
There is no tutorial here, but the absence of it feels justified given the context and theme of the game. There are not really even any hints about the items or the mechanics of the game, but rather the player has to figure it all out as the game unfolds. The oppressive feeling of war ties into this uncertainty and what is or is not important adds to the general atmosphere in a somewhat justifiable way. Lack of knowledge as a dynamic, then, becomes an understandable design choice that can enhance the experience.
\paragraph{Takeaways:}
Lack of knowledge can be used as a game dynamic, i.e. not giving out information even if it was possible to do so. This is also very much present in the general design of the Dark Souls games.

\subsection{Hitman (IO Interactive, 2016)}
\paragraph{Genre: Action / Stealth \\ Separate tutorial: Yes \\}
The game starts automatically with a tutorial mission, that walks the player through a simple mission that teaches the necessary mechanics and controls in order for the player to get a good grasp on available options. There are many different ways to complete a mission, but the controls required can be applied to different things in the same way, i.e. there is an interaction button that behaves in a different way depending on the context. It feels like an introductory mission such as this, that essentially holds the player's hand, is a good way to approach a game like this, which has more possible controls and interactions than just moving and shooting. Necessary controls are presented on the screen as well as in narration, and sometimes in a proper context the game pauses and tells you to press a button to activate an action, e.g. to subdue a person. In a simplified form, the player essentially gets told to go here, press this button, then go there and press this button and so on, so that it becomes clear what kind of actions are available.
\paragraph{Takeaways:}
Go through a whole single mission applying other heuristic principles to show how the game is played (or variations of how it can be played) step by step.

\subsection{Binary Domain (Sega, 2012)}
\paragraph{Genre: Third-person shooter \\ Separate tutorial: No \\} 
The tutorial is embedded in the regular gameplay. In the start of the game the player is asked if they want to view the basic controls of shooting and moving. Movement of the player is limited during this section, as individual controls are viewed one after another. Information of which buttons to press is presented on the screen and in narration. Context-sensitive information comes up for the remainder of the game whenever the player has the option to take cover, pick up items etc. This is presented on the screen with an icon of which button to play and a visual representation of the action.
\paragraph{Takeaways:}
Use context-sensitive information throughout the game to show when certain actions are possible, e.g. being able to take cover behind an obstacle. Ask the player whether they want to view the basic controls at the beginning of the game or not.

\subsection{Minecraft: Pocket Edition (Android) (Mojang, 2016)}
\paragraph{Genre: Sandbox / Survial \\ Separate tutorial: No \\}
The lack of any kind of a tutorial or guidance in the beginning makes this a difficult game to approach. Although basic movement controls are displayed on the screen constantly, emulating gamepad controls, it is difficult to find out how to do other things, such as craft equipment or anything beyond the basic controls. Even so much that it is necessary to search for information online on how to approach the game. 
\paragraph{Takeaways:}
It is important to consider showing basic information about the game, no matter how simple it may feel to the developer or designer.

\subsection{Monument Valley (Android) (Ustwo Games, 2014)}
\paragraph{Genre: Puzzle \\ Separate tutorial: No \\}
The first puzzle of the game introduces the controls to the player. There are only two of them, but they are presented clearly in text on the screen. The player can either move the character by tapping the screen, or change the form of the play area by tapping and dragging a designated area. There may not be much to do mechanics-wise, but the information still feels very helpful and necessary, and doesn't take a lot of time to absorb.
\paragraph{Takeaways:}
Even simple mechanics should be presented in a clear and noticeable manner.

\subsection{New Star Soccer (Android) (New Star Games, 2012)}
\paragraph{Genre: Sports \\ Separate tutorial: No \\}
New Star Soccer essentially consists of a few minigames that combine into a football manager game, but the most playing is done in actual soccer matches. There is a separate practice mode that tells you how to kick the ball to first time you play it, and also during regular gameplay in the menus there are pop-ups when actions are tried for the first time.
\paragraph{Takeaways:}
Include a practise mode for basic mechanics if they are difficult enough to master.
Context-sensitive help can also come up after, not before, the player takes actions, so they can discover things at their own pace.

\subsection{Insurgency (New World Interactive, 2014)}
\paragraph{Genre: Tactical shooter \\ Separate tutorial: Yes \\}
The tutorial of Insurgency follows a similar path as the tutorial of SWAT 4. There is an optional training facility that the player can go to, and he or she then is guided through a shooting range and an obstacle course with an instructor talking to the player about what to do next and how. Visual cues on controls are also presented and actions are sequenced so that the tutorial will not proceed until the player does the required action. 
\paragraph{Takeaways:}
Even in a realistic setting, using visual cues in the environment can be justified in a tutorial to enhance the speed of learning. All aspects of realism do not necessarily have to extend to at least the learning environment of the game.

\subsection{Hammerwatch (Crackshell, 2013)}
\paragraph{Genre: Hack and slash / Dungeon crawler \\ Separate tutorial: No \\}
There is not a tutorial as such, but rather buttons with question marks littered around the map. When the player activates these buttons, tips are presented in text on the screen. However, they refer to the more high-level mechanics such as what happens when the player goes to another floor or how vendors work. No help on e.g. the controls is presented; it seems to be assumed that players are familiar with the basic control scheme of a game such as this one.
\paragraph{Takeaways:}
The tutorial can also be about what the character can or should do, not how.




