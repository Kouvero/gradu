\appendix
\chapter{General heuristics}

\begin{center}
\begin{tabularx}{\textwidth}{|l|X|}
	\hline
	1. & Easy to learn, hard to master. \\ \hline
	2. & The goals are clear. \\ \hline	
	3. & The skills needed to attain goals are taught early enough to play or use later, or right before the new skill is needed. \\ \hline
	4. & The first ten minutes of play and player actions are painfully obvious and should result in immediate and positive feedback for all types of players. \\ \hline
	5. & Player does not need to read the manual or documentation to play. \\ \hline
	6. & Player does not need to access the tutorial in order to play. \\ \hline
	7. & Status score Indicators are seamless, obvious, available and do not interfere with game play. \\ \hline
	8. & Game provides feedback and reacts in a consistent, immediate, challenging and exciting way to the players’ actions. \\ \hline
	9. & Provide appropriate audio/visual/visceral feedback (music, sound effects, controller vibration). \\ \hline
	10. & Player is given controls that are basic enough to learn quickly, yet expandable for advanced options for advanced players. \\ \hline
	11. & Player interruption is supported, so that players can easily turn the game on and off and be able to save the games in different states. \\ \hline
	12. & Upon turning on the game, the player has enough information to begin play. \\ \hline
	13. & Players should be given context sensitive help while playing so that they are not stuck and need to rely on a manual for help. \\ \hline
	14. & All levels of players are able to play and get involved quickly and easily with tutorials, and/or progressive or adjustable difficulty levels. \\ \hline
	15. & Get the player involved quickly and easily. \\ \hline
	16. & The game should give hints, but not too many. \\ \hline
	17. & Allow users to skip non-playable and frequently repeated content. \\ \hline
	18. & Provide instructions, training, and help. \\ \hline
\end{tabularx}
\end{center}

\chapter{Heuristics for tutorials}
