\appendix
\chapter{Relevant heuristics related to tutorials in the bibliography}

\begin{center}
\begin{tabularx}{\textwidth}{|l|X|}
	\hline
	1. & Easy to learn, hard to master. \\ \hline
	2. & The goals are clear. \\ \hline	
	3. & The skills needed to attain goals are taught early enough to play or use later, or right before the new skill is needed. \\ \hline
	4. & The first ten minutes of play and player actions are painfully obvious and should result in immediate and positive feedback for all types of players. \\ \hline
	5. & Player does not need to read the manual or documentation to play. \\ \hline
	6. & Player does not need to access the tutorial in order to play. \\ \hline
	7. & Status score indicators are seamless, obvious, available and do not interfere with game play. \\ \hline
	8. & Game provides feedback and reacts in a consistent, immediate, challenging and exciting way to the players’ actions. \\ \hline
	9. & Provide appropriate audio/visual/visceral feedback (music, sound effects, controller vibration). \\ \hline
	10. & Player is given controls that are basic enough to learn quickly, yet expandable for advanced options for advanced players. \\ \hline
	11. & Player interruption is supported, so that players can easily turn the game on and off and be able to save the games in different states. \\ \hline
	12. & Upon turning on the game, the player has enough information to begin play. \\ \hline
	13. & Players should be given context sensitive help while playing so that they are not stuck and need to rely on a manual for help. \\ \hline
	14. & All levels of players are able to play and get involved quickly and easily with tutorials, and/or progressive or adjustable difficulty levels. \\ \hline
	15. & Get the player involved quickly and easily. \\ \hline
	16. & The game should give hints, but not too many. \\ \hline
	17. & Allow users to skip non-playable and frequently repeated content. \\ \hline
	18. & Provide instructions, training, and help. \\ \hline
\end{tabularx}
\end{center}

\chapter{Final heuristics identified for tutorials}

\begin{center}
	\begin{tabularx}{\textwidth}{|l|X|}
		\hline
		1. & The tutorial should reflect the intended gameplay experience. \\ \hline
		2. & The tutorial should reflect the pacing of the gameplay and not introduce mechanics before they are accessible in-game. \\ \hline
		3. & There can be multiple tutorials for different aspects of the gameplay. \\ \hline
		4. & Controls should be taught interactively when possible. \\ \hline
		5. & Learning through example is a possibility. \\ \hline
		6. & Using a training opponent before the actual game is a possibility. \\ \hline
		7. & Sometimes it is good to limit the possible controls to the ones being currently taught and pace the tutorial accordingly. \\ \hline
		8. & Teach applications and combinations of the basic controls that go beyond the basic control scheme. \\ \hline
		9. & Using a separate video to describe the usual game elements and dynamics is a possibility. \\ \hline
		10. & A short tutorial for a complex game can still be good. \\ \hline
		11. & Use different visual and audible ways to present information and color code keywords. \\ \hline
		12. & Have an option to turn off in-game tutorials. \\ \hline
		13. & Real-world scenarios can be copied to mimic training sessions in applicable genres. \\ \hline
		14. & Consider not showing all possible information depending on the setting; lack of knowledge can serve as a game dynamic. \\ \hline
		15. & Completing a whole mission step by step in an applicable genre can be helpful. \\ \hline
		16. & Use context-sensitive information throughout the game. \\ \hline
		17. & Do not underestimate the importance of a tutorial even in a simple game. \\ \hline
		18. & Present simple things in a clear and noticeable manner. \\ \hline
		19. & Consider a separate practise or free mode that can be used at the player's own pace. \\ \hline
		20. & Realism is not an excuse to not have additional visual elements in a tutorial setting. \\ \hline
		21. & The tutorial can also be about what the character can or should do, not just how. \\ \hline
		22. & The tutorial should not be made unnecessarily difficult. \\ \hline
		23. & The goals of the tutorial should be clear at all times. \\ \hline
		24. & The tutorial should begin with the most common interaction in the game. \\ \hline
		25. & The player should be able to skip the tutorial but not by accident. \\ \hline
		26. & The tutorial should give feedback of the player's required actions with visual and audible elements, and controller vibration if possible. \\ \hline
		27. & The player should be able to save the game during the tutorial. \\ \hline
	\end{tabularx}
\end{center}