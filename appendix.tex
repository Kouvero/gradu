\appendix
\chapter{Final heuristics identified for tutorials}

\begin{center}
	\begin{tabularx}{\textwidth}{|l|X|}
		\hline
		1. & The tutorial should reflect the intended gameplay experience. \\ \hline
		2. & The tutorial should reflect the pacing of the gameplay and not introduce mechanics before they are accessible in-game. \\ \hline
		3. & There can be multiple tutorials for different aspects of the gameplay. \\ \hline
		4. & Controls should be taught interactively when possible. \\ \hline
		5. & Learning through example is a possibility. \\ \hline
		6. & Using a training opponent before the actual game is a possibility. \\ \hline
		7. & Sometimes it is good to limit the possible controls to the ones being currently taught and pace the tutorial accordingly. \\ \hline
		8. & Teach applications and combinations of the basic controls that go beyond the basic control scheme. \\ \hline
		9. & Using a separate video to describe the usual game elements and dynamics is a possibility. \\ \hline
		10. & A short tutorial for a complex game can still be good. \\ \hline
		11. & Use different visual and audible ways to present information and color code keywords. \\ \hline
		12. & Have an option to turn off in-game tutorials. \\ \hline
		13. & Real-world scenarios can be copied to mimic training sessions in applicable genres. \\ \hline
		14. & Consider not showing all possible information depending on the setting; lack of knowledge can serve as a game dynamic. \\ \hline
		15. & Completing a whole mission step by step in an applicable genre can be helpful. \\ \hline
		16. & Use context-sensitive information throughout the game. \\ \hline
		17. & Do not underestimate the importance of a tutorial even in a simple game. \\ \hline
		18. & Present simple things in a clear and noticeable manner. \\ \hline
		19. & Consider a separate practise or free mode that can be used at the player's own pace. \\ \hline
		20. & Realism is not an excuse to not have additional visual elements in a tutorial setting. \\ \hline
		21. & The tutorial can also be about what the character can or should do, not just how. \\ \hline
		22. & The tutorial should not be made unnecessarily difficult. \\ \hline
		23. & The goals of the tutorial should be clear at all times. \\ \hline
		24. & The tutorial should begin with the most common interaction in the game. \\ \hline
		25. & The player should be able to skip the tutorial but not by accident. \\ \hline
		26. & The tutorial should give feedback of the player's required actions with visual and audible elements, and controller vibration if possible. \\ \hline
		27. & The player should be able to save the game during the tutorial. \\ \hline
	\end{tabularx}
\end{center}